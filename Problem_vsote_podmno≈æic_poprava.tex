\documentclass[12pt]{article}
 \usepackage[utf8]{inputenc}
   \usepackage[slovene]{babel}
\usepackage{amsthm,amsfonts,amsmath,amssymb,url}
\usepackage{hyperref}
\usepackage{listings}
\usepackage{xcolor}
\usepackage{graphicx}
\usepackage{pifont}
\usepackage{ bbding }

% \usepackage[table,xcdraw]{xcolor} 
\graphicspath{ {./images/} }

\definecolor{codegreen}{rgb}{0,0.6,0}
\definecolor{codegray}{rgb}{0.5,0.5,0.5}
\definecolor{codepurple}{rgb}{0.58,0,0.82}
\definecolor{backcolour}{rgb}{0.95,0.95,0.92}

\lstdefinestyle{mystyle}{
    backgroundcolor=\color{backcolour},   
    commentstyle=\color{codegreen},
    keywordstyle=\color{magenta},
    numberstyle=\tiny\color{codegray},
    stringstyle=\color{codepurple},
    basicstyle=\ttfamily\footnotesize,
    breakatwhitespace=false,         
    breaklines=true,                 
    captionpos=b,                    
    keepspaces=true,                                 
    showspaces=false,                
    showstringspaces=false,
    showtabs=false,                  
    tabsize=2
}

\lstset{style=mystyle}

\textheight 210 true mm
\textwidth 146 true mm
\voffset=-17mm
\hoffset=-13mm

\newtheorem{Izrek}{{\sc Izrek}}[section]
\newtheorem{Trditev}[Izrek]{{\sc Trditev}}
\newtheorem{Lema}[Izrek]{{\sc Lema}}
\newtheorem{Posledica}[Izrek]{{\sc Posledica}}
\newtheorem{Definicija}[Izrek]{{\sc Definicija}}
\newtheorem{Domneva}[Izrek]{{\sc Domneva}}
\newtheorem{Zgled}[Izrek]{{\sc Zgled}}
\newtheorem{Opomba}[Izrek]{{\sc Opomba}}

\def\theIzrek{{\rm \arabic{section}.\arabic{Izrek}}}

\newenvironment{izrek}{\begin{Izrek}\sl}{\end{Izrek}}
\newenvironment{trditev}{\begin{Trditev}\sl}{\end{Trditev}}
\newenvironment{lema}{\begin{Lema}\sl}{\end{Lema}}
\newenvironment{posledica}{\begin{Posledica}\sl}{\end{Posledica}}
\newenvironment{definicija}{\begin{Definicija}\rm }{\end{Definicija}}
\newenvironment{domneva}{\begin{Domneva}\rm }{\end{Domneva}}
\newenvironment{zgled}{\begin{Zgled}\rm}{\end{Zgled}}
\newenvironment{opomba}{\begin{Opomba}\rm}{\end{Opomba}}

\newenvironment{dokaz}[1][{\sc Dokaz}]{\begin{proof}[#1]\renewcommand*{\qedsymbol}{\(\square\)}}{\end{proof}}

\newcommand{\Mod}[1]{\hbox{ (mod } #1)}
\newcommand{\xmark}{\ding{55}}%
\begin{document}

\thispagestyle{empty}
\begin{center}
\begin{Large}
{\bf Problem vsote podmnožic}
\end{Large}
\\[5mm]
\begin{large}
Pisni izdelek pri predmetu {\em Računalništvo 1}
\\[5mm]
{\sc Tit Arnšek}
\\[10mm]
31.~december, 2021
\end{large}

\end{center}

\newpage

\tableofcontents
\newpage
\listoffigures

\newpage
\section{Uvod}
Problem vsote podmnožic je v računalništvu problem odločanja(tj. problem, na katerega
lahko odgovorimo z da/ne).\newline
Pri problemu imamo podano množico nenegativnih celih števil in pozitivno  celo število. Števila v množici so neurejena.  
Zanima nas, ali obstaja kakšna podmnožica dane množice, katere vsota elementov se sešteje v  število, ki smo ga podali.
Zanimalo nas ne bo, koliko je takšnih podmnožic in katera števila vsebujejo, ampak samo, ali obstaja kakšna ali ne. \newline
Problem lahko definiramo tudi za multimnožice (dovolimo ponavljanje elementov), ampak bomo uporabljali le "prave" množice. 
Načeloma pa se vsi algoritmi, ki jih bomo obravnavali, obnašajo enako, tudi če uporabimo multimnožice.

 Primer:

\[ Mn = \{1,3,6,7,33,8,15\}; vsota = 16\]
 $$Resitev: \{3,6,7\},\{1,7,8\},\{1,15\}  \colorbox{green}{\checkmark}$$ 
V zgornjem primeru vdimo da takšna podmnožica obstaja. 

 $$Mn = \{1,3,6,7,33,8,15\}; vsota = 33$$
 $$Resitev: \{33\} \colorbox{green}{\checkmark}$$ 
Tudi v tem primeru obstaja takšna podmnožica

 $$Mn = \{1,3,6,7,33,8,15\}; vsota = 5   $$
 $$Resitev: \colorbox{red}{\text{\sffamily X}}$$ 
 V tem primeru pa takšne podmnožice ni.
 \newline

 V zgornjih primerih je bilo precej enostavno ugotoviti, ali takšna
 podmnožica obstaja. Kaj pa če imamo v podani množici veliko števil?\newline
Vzemimo spodnjo množico in vsoto:

 \[Mn = \{86, 50, 49, 42, 83, 8, 90, 56, 6,26, 73, 8, 47, 21, 4, 24,\]
 \[15, 57, 61, 100, 66, 86, 68,  99, 40, 22, 43, 32, 89, 92\}; vsota = 1100\]\newline
V tem primeru pa ni takoj vidno, ali takšna podmnožica obstaja ali ne.
Izkaže se, da takšna podmnožica obstaja, postopke, ki pripeljejo do tega spoznanja, pa
bomo opisali kasneje.\newline
V nadaljevanju so opisani trije algoritmi, s katerimi 
rešimo dani problem.



\section{Eksponentni Algoritmi}
Prva dva algoritma, ki sta opisana, imata eksponentno časovno zahtevnost.
 

  \subsection{Brute - force}
  \label{sec:brute}
  Prvi algoritem je zasnovan na strategiji uporabe grobe sile.
  Pri tej metodi moramo poiskati vse možne podmnožice in izračunati njihove vsote.
  \subsubsection{Implementacija kode}  
  Sam algoritem bi v pythonu napisali na sledeči način:
  \label{sec:koda}
 \begin{lstlisting}

  def vsota_podmnozic_brute_force(S, k):
    '''
        brute force metoda za iskanje vseh podmnozic mnozice 'S',
        ki imajo vsoto elementov enako 'k'. V primeru, 
        da taksne podmnozice ni, vrne prazno tabelo
    '''
    podmnozice =[set()]
    resitev = []
    for st in S:
        nove_podmnozice = []
        for stara in podmnozice:
            novi = stara.copy()
            novi.add(st)
            if sum(novi) == k:# najdena resitev
                resitev.append(novi)
            #dodaj k moznim podmnozicam
            nove_podmnozice.append(novi)
            nove_podmnozice.append(stara)
        podmnozice = nove_podmnozice
    return resitev
  \end{lstlisting}
  \begin{opomba}
    Zgornja funkcija nam vrne tabelo vseh podmnožic z vsoto enako $k$, ali pa nam vrne prazni seznam,
    če takšna množica ne obstaja.
  \end{opomba}
  



  \subsubsection{Časovna zahtevnost}
  Če imamo množico z $n$ elementi, je vseh možnih podmnožic $2^n$. \newline
  Za vsako podmnožico moramo še izračunati njeno vsoto, kar nam kvečjemu
  vzame $n$ korakov. \newline
  V najslabšem primeru(ko takšna podmnožica ne obstaja) je časovna zahtevnost $O(2^n \cdot n)$.
  Izkaže se, da je tudi pričakovana časovna zahtevnost $O(2^n \cdot n)$(\url{https://en.wikipedia.org/wiki/Subset_sum_problem#Inclusion%E2%80%93exclusion},2022).


 \subsection{Horowitz and Sahni}
 Spodaj opisani algoritem sta prvič objavila 
 Ellis Horowitz\footnote{Ellis Horowitz je ameriški računalničar, 
 profesor računalništva in elektrotehnike na Univerzi v Južni Kaliforniji (\url{https://en.wikipedia.org/wiki/Ellis_Horowitz}, 2021).} 
 in Sartaj Kumar Sahni.\footnote{Sartaj Kumar Sahni je računalniški znanstvenik v 
 Združenih državah Amerike in je eden od pionirjev na področju podatkovnih 
 struktur. Je profesor na oddelku za računalništvo, informatiko 
 in inženiring na Univerzi Florida(\url{https://en.wikipedia.org/wiki/Sartaj_Sahni}, 2021).}
 Glavna ideja je, da algoritem pohitrimo s tem, da podano množico razdelimo na dve enako veliki podmnožici, na katerih
 zgeneriramo vse možne podmnožice. Podmnožice iz obeh delov nato kombiniramo tako, da se 
 določenim kombinacijam izognemo. Na ta način ne preverimo vseh možnih podmnožic, kot
 smo jih sicer pregledali z \textit{brute - force} algoritmom.
\newline
 Naredimo naslednje korake:

 \begin{itemize}
  \item Razdelitev množice:\newline
  \begin{lstlisting}

    Mn= {3,1,6,33,8,15,7} Vsota = 27
    prva = {3,1,6,33} druga = {8,15,7}
  \end{lstlisting}
  % $Mn= \{3,1,6,33,8,15,7\}$
  % $prva = \{3,1,6,33\}$ $druga = \{8,15,7\}$.
  \begin{opomba}
    Če je število elementov v množici liho, potem v prvo podmnožico damo eno
    število več kot v drugo.
  \end{opomba}
  \item Za vsako množico, ki ima $n/2$ števil(ali $n/2 + 1$) 
        zgeneriramo seznam parov možnih podmnožic in njihovih elementov. Teh parov imata vsaka po $2^{n/2}$. 
  \begin{lstlisting}

    prva_podm = [({},0),({3},3),({1},1),...,({3,1,6,33},43)]
    druga_podm = [({},0),({8},8),({15},15),...,({8,15,7},30)]
  \end{lstlisting} 


        % $$prve\_podmnozice = [{},{}...] druge\_podmnozice = [{},{}....]$$
  \item Oba seznama uredimo glede na vsoto podmnožic, ki so v seznamu.
  %  $$prve_podmnožice = [{},{}....]druge_podmnožice = [{},{}....]$$
  \item Če smo že našli kakšno podmnožico z ustrezno vsoto, potem smo z algoritmom končali,
  sicer pa gremo na naslednji korak.
  \item Prvi kazalec damo na začetek prvega seznama, drugi kazalec pa na konec drugega.
  \item Če je vsota obeh vsot podmnožic, na katere kažeta kazalca, večja od iskane vsote, potem 
  drugi kazalec premaknemo za eno v levo po drugem seznamu, sicer pa premaknemo prvi kazalec za ena v desno v prvem seznamu. \newline
  Ta korak delamo, dokler ne pridemo do želene vsote(našli smo podmnožico. Ta je sestavljena iz množic,
  na katere kažeta kazalca) ali pa pridemo s kazalci do konca seznama(nismo našli podmnožice). 
\end{itemize}

Na \hyperref[sec:slika1]{Sliki 1} si oglejmo primer. Kazalca sta predstavljena z modro oziroma rdečo črto. Modra barva označuje, da 
je kazalec ostal na istem mestu v tabeli, rdeča črta pa označuje, da se je kazalec premaknil bodisi v levo bodisi v desno) 
\newline
\begin{figure}[h] \label{sec:slika1}
  \caption{Primer Horowitz Sahni}
  \includegraphics[width = 14.6cm]{p1}
  \newline
  \newline
  Ker smo dobili želeno vsoto, smo z algoritmom zaključili.
  Podmnožico, katere vsota elementov je enaka dani vsoti, pa dobimo tako,
  da obe podmnožici, na kateri kažeta kazalca, združimo.\newline
  V zgornjem primeru bi to bila podmnožica $\{1,3,8,15\}$.
\end{figure}
\begin{opomba}
  Pri pregledu pregledamo vse možnosti, ki bi lahko imele pravino vsoto:
  - če se podmnožica z iskano vsoto nahaja v eni izmed podmnožic, potem algoritem končamo že 
  pred kombiniranjem podmnožic.
  - Naj bo levi kazalec levo od $m_1$ desni pa desno od $m_2$.
  Recimo, da je trenutna vsota manjša. To pomeni, da levi kazalec premaknemo v desno. Če pridemo na $m_1$ je vsota zagotovo večja, zato premikamo
  desni kazalec... 
  - če množica obstaja, potem je sestavljena iz množice $m_1$ iz $prva\_podm$ z vsoto $k_1$ in 
  množice $m_2$ iz  $druga\_podm$ z vsoto $k_2$. Na začetku vemo, da je levi kazalec levo od $m_1$ desni pa desno od $m_2$.
  Slej ko prej pride eden izmed kazalcov na $m_1$ oz $m_2$. Recimo, da je levi kazalec prišel prej na $m_1$. V tem koraku vemo, da je desni kazalec
  še vedno desno od $m_2$. Vsota je potem zagotovo večja od iskane vsote, zato desni kazalec premaknemo v levo\dots
  pride kazalec na $m_1$, vemo, da je bila prejšnja dobljena vsota manjša, ker smo prestavili levi kazalec. To pomeni, da je bil kazalec 
  na desni na množici z vsoto, ki je večja od $k_2$. Sedaj je skupna vsota večja, zato moramo prestaviti desni kazalec v levo.
  Prestavljali ga bomo toliko časa, dokler ne bo vsota zopet manjša oziroma enaka.
  To bo natanko tedaj, ko bomo prišli do $m_2$. Podobno bi lahko trdili za primer, če gremo najprej na $m_2$\dots

  

\end{opomba}
POPRAVI!!! UTEMELJI, ZAKAJ smo s tem res preverili VSE podmnožice, ki bi lahko imele tako vsoto!)


\newpage
 \subsubsection{Implementacija kode}
 \begin{lstlisting}
  def vsota_podmnozic_horowitz_sahni(S,k):
    '''
        Metoda horowitz-sahni, ki nam vrne prvo podmnozico 
        iz mnozice 'S', ki ima vsoto elementov enako 'k'. 
        V primeru, da taksne podmnozice ni, vrne prazno tabelo
    '''
    #Mnozico S razdelimo na dve podmnozici enakih velikosti
    prva = S[:len(S)//2]
    druga = S[(len(S)//2):]
    #Zgeneriramo vse podmnozice in njihove vsote od prve in druge
    #podmnozice.Vsako mnozico nato uredimo po vsotah mnozic
    prva_podm = sorted(vse_podmnozice_z_vsoto(prva),key = lambda x: x[1])
    druga_podm = sorted(vse_podmnozice_z_vsoto(druga),key = lambda x: x[1])
    kazalec1 = 0
    kazalec2 = len(druga_podm) - 1
    while kazalec1 < len(prva_podm) and kazalec2 >= 0:
        #sestejemo vsote
        delna = prva_podm[kazalec1][1]+ druga_podm[kazalec2][1] 
        if delna == k:#nasli smo podmnozico, ki je sestavljena iz
                      #prva_podm[kazalec1] in druga_podm[kazalec2]
            return prva_podm[kazalec1][0].union(druga_podm[kazalec2][0])
        if delna < k:
            kazalec1 += 1
        else:
            kazalec2 -= 1
    return []
\end{lstlisting}

\begin{opomba}
  Pri zgornji implementaciji je uporabljena funkcija \newline \textit{vse\_podmnozice\_z\_vsoto}, 
  ki iz podane množice ustvari tabelo parov vseh podmnožic in vsot.
\end{opomba}
\subsubsection{Časovna zahtevnost}
  Zgornji algoritem je prav tako eksponentne časovne zahtevnosti. Bolj natančno, časovna zahtevnost
  je $O(2^{n/2} \cdot n/2)$. Prvi faktor($2^{n/2}$) sledi iz tega, da moramo zgenerirati $2^{n/2}$ podmnožic
  (moramo jih zgenreirati $2 \cdot 2^{n/2} = 2^{n/2 + 1}$), drugi faktor $n/2$ pa dobimo zaradi računanja vsote vsake podmnožice
  (sešteti moramo kvečjemu $n/2$ števil). Zadnji del algoritma, kjer prestavljamo kazalce in primerjamo vsote, 
  ima časovno zahtevnost $O(2^{n/2})$, ker moramo nareditit kvečjemu $2 \cdot 2^{n/2}$ premikov. 
 \newline

  Poleg zgoraj omenjenih algoritmov obstajata še dva pogosto uporabljena algoritma, kjer je časovna zahtevnost eksponentna.
  To sta algoritem \textit{Schroeppel in Shamir} ter algoritem   \textit{Howgrave-Graham in Joux}. 
  Pri prvem množico razdelimo na 4 dele, vendar je v nadaljevanju postopek malo bolj kompleksen kot pri algoritmu
  \textit{Horowitz and Sahni}. Pri prvem algoritmu je časovna zahtevnost ${O(2^{n/2}  \cdot  (n/4))}$ pri drugem pa
  $O(2^{0.337n})$ (\url{https://en.wikipedia.org/wiki/Subset_sum_problem#Exponential_time_algorithms}, 2021).



  \section{Rešitev z dinamičnim programiranjem}
  Pri dinamičnem programiranju sta običajno možna dva pristopa: pristop naprej(\textit{top-down approach}) in pristop nazaj(\textit{bottom-up approach}).
  Pri prvem gre za uporabo rekurzije z uporabo memoizacije, pri drugem pa lahko problem rešimo iterativno,
  pomagamo pa si s pomočjo tabele. V nadaljevanju bo predstavljena iterativna rešitev problema.



 \subsection{Iterativna rešitev}
 Ideja algoritma je, da s sprotnim dodajanjem števil, preverjamo katere so vse možne vsote, ki jih
 lahko dobimo z že dodanimi števili.\newline
 Pri algoritmu si pomagamo s tabelo dimenzije $(n+1) \cdot (k+1)$, kjer je
 $n$ velikost podane množice, $k$ pa vsota, ki jo želimo dobiti. Nad vsak stolpec tabele bomo napisali število (od 0 do $k$), 
 levo od vsake vrstice pa števila, ki jih imamo v množici. Tabelo bomo polnili iz vrha
 proti dnu, same vrednosti v tabeli (True/False) pa nam bodo povedale, katere vsote lahko naredimo s 
 števili, navedenimi v levem stolpcu do te vrstice. 
 \newline
Primer: \label{sec:s1}
\begin{figure}[h]
  \caption{Primer tabele}
  \label{sec:s2}
\includegraphics[width=14.6cm]{s1}
\centering
\end{figure}\newline
Iz zadnje vrstice s \hyperref[sec:s2]{slike 2} lahko razberemo, da iz števil $2,3$ in $6$ lahko naredimo vsote $0,2,3,5,6,8,9$ in $11$.
Iz tretje vrstice lahko razberemo, da iz števil $2$ in $3$ lahko skonstruiramo množico, katere vsota števil 
je enaka $0,2,3$ ali $5$ itd.  
\newline Tabela je velikosti $(n+1)  \cdot  (k+1)$ in ne $n  \cdot  (k+1)$, ker v prvi vrstici
'umetno' dodamo število 0. \newpage
\subsubsection{Kako napolniti tabelo?}
Tabelo s \hyperref[sec:s2]{slike 2} lahko zlahka sestavimo peš. 
Oglejmo si, kako lahko postopek naredimo s pomočjo algoritma. 
\newline
\label{sec:s2}
\begin{figure}[h]
  \caption{Primer tabela 2}
  \includegraphics[width=14.6cm]{s2}
  Kako vemo, katere vsote je možno dobiti, ko vstavimo nov element v tabelo?
Opazimo lahko tri stvari:
\end{figure}
\begin{itemize}
  \item če smo s prejšnimi števili lahko dobili neko vsoto, potem jo  lahko tudi
  po tem, ko dodamo novo število v tabelo,
  \item število, ki ga vstavimo v tabelo, ne more vplivati na vsote, ki so manjše od dodanega števila
  (to velja zato, ker smo predpostavili, da so v množici le nenegativna števila),
  \item če sestavljamo vrstico za število $n$, potem lahko vsoto $k$ dobimo le v primeru, če smo s prejšnimi
  števili že lahko dobili to vsoto(1. točka) ali pa, če smo s prejšnimi števili dobili vsoto $k-n$.
\end{itemize}
Če upoštevamo zgornje trditve, smo dobili algoritem, s katerim zapolnimo vsa mesta v tabeli.
\newline Denimo, da sestavljamo vrstico $i$, kjer želimo dodati število $n$. Potem:
\begin{itemize}
  \item na vsa mesta v tabeli $A$, kjer je vsota manjša od  
  števila, ki ga obravnavamo, prepišemo vrednosti zgornje vrstice iz tabele:
  \[A(i,j) = A(i-1,j),\]
  \item za mesta, kjer je vsota večja ali enaka obravnavanemu številu, pa pogledamo 
  vrednosti iz zgornje vrstice: Če obravnavamo število $k$ za vsoto $j$, potem preverimo ali je možno
  s prejšnimi števili dobiti bodisi vsoto $j$ bodisi vsoto $j-k$. 
  \[A(i,j) = A(i-1,j) \text{ or }  A(i-1,j-k),\] kjer smo v $i$-ti vrstici dodali število $k$.
\end{itemize}
Zgoraj prikazano \hyperref[sec:s2]{tabelo} bi dopolnili na sledeči način:\newline
\begin{figure}[h]
  \caption{Sestavljanje tabele 2}  
  \includegraphics[width=14.6cm]{s3}
  \centering
\end{figure}\newline

Tabele nam ni treba dopolniti do konca. 
Postopek lahko kočamo že prej, ko v zadnjem stolpcu dobimo vrednost $True$. To pomeni, da smo našli neko 
kombinacijo števil, s katerimi lahko dobimo želeno vsoto. Če takšne kombinacije ni,
potem bomo zapolnili celo tabelo, vrednost na desnem spodnjem mestu v tabeli pa bo $False$.\newline
Poglejmo si zgornji algoritem na primeru:\newline
$$Mn = \{2,3,15,10,6,7\}, vsota = 11$$

\begin{figure}[h]
  \caption{Primer dinamične rešitve: generiranje tabele}  
  \includegraphics[width=14.6cm]{s4}
  Ustvarili smo tabelo velikosti $7 \cdot 12$, kjer smo levo od vsake
  vrstice zapisali eno od števil iz podane množice, nad vsak stolpec pa smo 
  zaporedoma zapisali vsa števila od 0 do 11. Levo od prve vrstice v tabeli zapišemo 
  število 0, vse vrednosti v prvi vrstici (razen v prvem stolpcu) pa zapolnemo s \textit{False}(F).
\end{figure}\newpage


\begin{figure}[h]
  \caption{Primer dinamične rešitve: 2. vrstica}  
  \includegraphics[width=14.6cm,height=4.6cm]{s5}
  Pri obravnavi druge vrstice pogledamo, katere vsote lahko dobimo s prejšnimi števili(0) in 
  številom 2. Na mestu v prvem in drugem stolpcu v tej vrstici prepišemo vrednosti iz zgornje(prve) vrstice,
  za vrednosti na ostalih mestih pa pogledamo mesta bodisi zgoraj bodisi zgoraj in za dve mesti levo.
\end{figure}
Po nekaj korakih smo napolnili tabelo, kot je vidno na sliki 5.
\begin{figure}[h]
  \caption{Primer dinamične rešitve 5. vrstica} 
   \includegraphics[width=14.6cm]{s8}
   V tej vrstici bi se lahko  spremenila vrednost zgolj v predzadnjem in zadnjem stolpcu.
   V predzadnjem stolpcu nastavimo vrednost na \textit{True}(T), ker smo s prejšnjimi števili 
   lahko dobili vsoto $0 (10-10)$, v zadnjem stolpcu pa vrednost ostane \textit{False}(F), saj s prejšnimi 
   števili ne moremo dobiti niti $1 (11 - 10)$ niti 11.

  \end{figure}

  \begin{figure}[h]
  \caption{Primer dinamične rešitve 6. vrstica}
  \includegraphics[width=14.6cm]{s9}
  
\end{figure}
\newpage
V zadnjem koraku smo v zadnjem stolpcu dobili vrednost True. To pomeni,
da obstaja podmnožica množice $\{2,3,15,10,6,7\}$, katere elementi se seštejejo v $11$.
\newpage
\subsubsection{Implementacija kode}
\begin{lstlisting}
def dinamicna(set, sum):
  n = len(set)
  #Naredimo tavelo velikosti n+1 * k+1
  subset =([[False for i in range(sum + 1)] for i in range(n + 1)])
  #Prvi stolpec nastavimo na true
  for i in range(n + 1):
      subset[i][0] = True
  #gremo od 2. vrstice naprej, ker v prvi smo dodali 'umetno' 0
  for i in range(1, n + 1):
      for vsota in range(1, sum + 1):
          if vsota < set[i-1]: #stevilo je vecje od dane vsote, zato samo pogledamo zgornje vrednosti 
              subset[i][vsota] = subset[i-1][vsota] 
          if vsota >= set[i-1]:
              subset[i][vsota] = (subset[i-1][vsota] or #PREVERIMO CE JE ZGORNJI TRUE
                              subset[i - 1][vsota-set[i-1]]) 
                # GREMO GOR IN LEVO KOLIKOR je DANO STEVILO
                    #pregledamo posebej se za zadnji stolpec
      vsota = sum
      if vsota < set[i-1]: #stevilo je vecje od dane vsote, zato samo pogledamo zgornje vrednosti 
          subset[i][vsota] = subset[i-1][vsota] 
      if vsota >= set[i-1]:
          subset[i][vsota] = (subset[i-1][vsota] or #PREVERIMO CE JE ZGORNJI TRUE
                          subset[i - 1][vsota-set[i-1]]) # GREMO GOR IN LEVO KOLIKOR JE DANO STEVILO
          if subset[i][vsota]:
              return True #dobili smo neko kombinacijo. Lahko koncamo
  #prisli smo do konca... v zadnjem stolpcu je False!
  return False
\end{lstlisting}

\subsubsection{Časovna zahtevnost}
Pri algoritmu moramo v najslabšem primeru (ko podmnožica ne obstaja) izpolniti vsa mesta
v tabeli. Teh mest je $n \cdot k$, kjer je $n$ velikost množice, $k$ pa iskana vsota. Na vsakem mestu naredimo 
kvečjemu 2 primerjavi, zato je časovna zahtevnost $O(n \cdot k)$. 

\subsubsection{Možne izboljšave}
Za obravnavanje poljubne vrstice v tabeli potrebujemo zgolj zgornjo vrstico. 
Zato si ni potrebno shranjevati celotne tabele, ampak si je dovolj zapomniti le zadnjo obravnavano
vrstico. \newline
Časovna zahtevnost se v tem primeru ne spremeni, spremeni pa se prostorska zahtevnost, ki
je v prejšnem primeru $O(n\cdot k)$, sedaj pa je le $O(k)$ (podatke si shranjujemo v dva seznama).


\section{Primerjava algoritmov}
\begin{table}[h]
  \begin{tabular}{|l|l|l|}
  \hline
                     & Horowitz and Sahni                                                               & \begin{tabular}[c]{@{}l@{}}Dinamično \\ programiranje\end{tabular}                    \\ \hline
  \textbf{PREDNOSTI} & \begin{tabular}[c]{@{}l@{}}Enostavno dobimo vse \\ možne podmnožice\end{tabular} & \begin{tabular}[c]{@{}l@{}}V večini primerov \\ hitrejši algoritem\end{tabular}       \\ \hline
  \textbf{SLABOSTI}  & Eksponentna časovna zahtevnost                                                   & \begin{tabular}[c]{@{}l@{}}Časovna zahtevnost \\ odvisna od želene vsote\end{tabular} \\ \hline
  \end{tabular}
  \end{table}
\begin{opomba}
V zgornjo tabeli ni algoritma \textit{brute - force}. Njegove lastnosti so podobne kot pri algoritmu
\textit{Horowitz and Sahni}, s tem da je še počasnejši.
\end{opomba}
\subsubsection{Grafična predstavitev}
Spodaj je prikazanih še nekaj grafičnih primerjav med algoritmi. Pri poizkusih sem za 
želeno vsoto vzel vrednost $k = 1000$. \newpage
\begin{figure}[h]
  \caption{Graf: primerjava HS in dinamična} 
  \centering 
  \includegraphics[height = 6.9cm]{g1}
\end{figure}
  Na zgornjem grafu se opazi, da algoritem \textit{Horowitz and Sahni} 
  'eksplodira' zelo hitro. Za rešitev z dinamičnim programiranjem je 
  za dane velikosti množic(do 30), računalnik potreboval 
  zanemarljivo malo časa.\newline
  Na spodnjem garfu pa opazimo, da \textit{Brute - force} algoritem eksplodira še bistveno hitreje(okoli $n = 20$).
  \begin{figure}[h]
    \centering 
  \caption{Graf: prirmerjava HS in BF}  
  \includegraphics[height = 6.9cm]{g2}
  
\end{figure}

\newpage
  \begin{figure}[]
  \caption{Graf: dinamična rešitev} 
  \centering
  \includegraphics[height = 6.9cm]{g3}\newline
\end{figure}
Za fiksno vsoto (v tem primeru $k = 1000$) narašča čas linearno, kar se ujema
  s trditvijo, da je časovna zahtevnost $O(n\cdot k)$. Če $k$ fiksiramo 
  je časovna zahtevnost linearna ($O(n)$).

\section{Uporaba}
Problem vsote podmnožic se lahko uporablja v računalništvu za preverjanje gesel.
Najpreprostejši način preverjanja bi bil, da bi program, ki preverja pravilnost identitete,
imel kopijo gesla shranjeno nekje v datoteki. Problem bi nastal, da vsak, ki bi videl
to datoteko bi s tem prišel do gesel ostaliih uporabnikov.
Nekateri sistemi uporabijo rešitev, da namesto da bi v datoteki shranjevali gesla(nize znakov)
shranjujejo število, ki predstavlja vsoto. \newline
Ko uporabnik venese niz zankov, program pretvori dano zaporedje znakov v neko podmnožico števil.
 Če smo vnesli pravo zaporedje znakov, bo program vrnil takšno podmnožico, katere vsota
je enaka tisti, ki je shranjena v datoteki. Na tak način naše geslo ni nikjer vidno.
(\url{http://www.math.stonybrook.edu/~scott/blair/Other_uses_subset_sum.html},2002)
POPRAVI!!!VIR
\newpage
\section{Viri}
\begin{itemize}
\item Subset Sum Problem | DP-25,  \newline
\url{https://www.geeksforgeeks.org/subset-sum-problem-dp-25/}, \newline
(ogled: 28.\ december 2021).
\item Subset Sum Problem Dynamic Programming,
\url{https://www.youtube.com/watch?v=s6FhG--P7z0&ab_channel=TusharRoy-CodingMadeSimple},\newline
(ogled: 28.\ december 2021).
\item Subset sum problem, Wikipedia, the free encyclopedia,\newline
\url{https://en.wikipedia.org/wiki/Subset_sum_problem} \newline
(ogled: 28.\ december 2021).
POPRAVI!!!DODAJ VIRE
\end{itemize}

  

\end{document}